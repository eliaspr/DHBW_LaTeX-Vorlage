% +------------------------------------+
% | Beginn des Konfigurationsbereiches |
% +------------------------------------+

\newcommand{\defTitel}{Titel}
\newcommand{\defAutor}{Autor}
\newcommand{\defAbgabedatum}{Abgabedatum}
\newcommand{\defArbeit}{T3-1000}
\newcommand{\defArbeitKategorie}{Projektarbeit} % Projektarbeit Bachelorarbeit Studienarbeit
\newcommand{\defStudiengang}{Angewandte Informatik}
\newcommand{\defHochschule}{Dualen Hochschule Baden-Württemberg Mosbach}
\newcommand{\defKurs}{MGH-INF19}
\newcommand{\defMatrikelnummer}{123456789}
\newcommand{\defBearbeitungszeitraum}{12 Wochen}
\newcommand{\defAusbildungsfirma}{Meine Firma GmBH}
\newcommand{\defAusbildungsfirmaOrt}{12345 Musterstadt}
\newcommand{\defBetreuer}{Max Mustermann}

% Folgende Zeile auskommentieren, falls kein Gutachter vorhanden ist (Gutachter i.d.R. nur Bachelorarbeit).
\newcommand{\defGutachter}{Max Musterbetreuer, M. Sc.}

% Folgende Zeile auskommentieren, um den "Entwurf"-Hinweis zu deaktivieren
\newcommand{\defEntwurf}

% Folgende Zeile einkommentieren, um alle Quellenangaben im Literaturverzeichnis anzuzeigen. 
% \newcommand{\defAlleQuellenZeigen}

% Folgende Zeile auskommentieren, um die bunten Rahmen um Hyperlinks, Referenzen etc. auszublenden
\newcommand{\defShowLinkBorders}

% +------------------------------------+
% |  Ende des Konfigurationsbereiches  |
% +------------------------------------+




%!TEX root = ../dokumentation.tex

\RequirePackage[l2tabu, orthodox]{nag} % weist in Commandozeile bzw. log auf veraltete LaTeX Syntax hin

\documentclass[%
    pdftex,
    oneside,             % Einseitiger Druck.
    12pt,                % Schriftgroesse
    parskip=half,        % Halbe Zeile Abstand zwischen Absätzen.
    headheight = 12pt,   % Höhe der Kopfzeile
    headsepline,         % Linie nach Kopfzeile.
    footsepline,         % Linie vor Fusszeile.
    footheight = 16pt,   % Höhe der Fusszeile
    abstracton,          % Abstract Überschriften
    DIV=calc,            % Satzspiegel berechnen
    BCOR=8mm,            % Bindekorrektur links: 8mm
    headinclude=false,   % Kopfzeile nicht in den Satzspiegel einbeziehen
    footinclude=false,   % Fußzeile nicht in den Satzspiegel einbeziehen
    listof=totoc,        % Abbildungs-/ Tabellenverzeichnis im Inhaltsverzeichnis darstellen
    toc=bibliography     % Literaturverzeichnis im Inhaltsverzeichnis darstellen
]{scrbook}  % Koma-Script report-Klasse, fuer laengere Bachelorarbeiten alternativ auch: scrbook

\usepackage[dvipsnames]{xcolor}
\usepackage{xstring}
\usepackage[utf8]{inputenc}
\usepackage[T1]{fontenc}
\usepackage[ngerman]{babel}
\usepackage[margin=2.5cm,foot=1cm]{geometry}
\usepackage[activate]{microtype}
\usepackage[onehalfspacing]{setspace}
\usepackage{makeidx}
\usepackage[autostyle=true,german=quotes]{csquotes}
\usepackage{longtable}
\usepackage[shortlabels]{enumitem}
\usepackage{graphicx}
\usepackage{pdfpages}
\usepackage{float}
\usepackage{array}
\usepackage{calc}
\usepackage[right]{eurosym}
\usepackage{wrapfig}
\usepackage{pgffor}
\usepackage[perpage, hang, multiple, stable]{footmisc}
\usepackage[printonlyused, nohyperlinks]{acronym}
\usepackage{listings}
\usepackage{siunitx}
\DeclareSIUnit\byte{B}
\usepackage{subcaption}
\usepackage{pgfplots, pgfplotstable}
\usepackage[section]{placeins}
\usepackage{amsmath}
\usepackage{amssymb}
\usepackage{makecell}
\usepackage{multirow}
\usepackage{diagbox}
\usepackage{icomma}
\pgfkeys{/pgf/number format/.cd,use comma}
\usepackage{caption}
\usepackage[textsize=scriptsize, loadshadowlibrary, shadow]{todonotes}
\captionsetup{justification   = raggedright,
              singlelinecheck = false}

\makeatletter
\AtBeginDocument{%
  \expandafter\renewcommand\expandafter\subsection\expandafter{%
    \expandafter\@fb@secFB\subsection
  }%
}
\makeatother

\usepackage{lmodern} % palatino oder goudysans, lmodern, libertine
\definecolor{LinkColor}{HTML}{00007A}
\definecolor{ListingBackground}{HTML}{FCF7DE}

\title{\defTitel}
\author{\defAutor}
\date{\defAbgabedatum}

\usepackage[%
    pdftitle={\defTitel},
    pdfauthor={\defAutor},
    pdfsubject={\defArbeit},
    pdfcreator={pdflatex, LaTeX with KOMA-Script},
    pdfpagemode=UseOutlines,
    pdfdisplaydoctitle=true, 
    pdflang={de}
]{hyperref}

\hypersetup{%
    colorlinks=false, 
    bookmarksnumbered=true
}
\usepackage{bookmark}

\addtokomafont{caption}{\small}

\usepackage[
    backend=biber,
    bibwarn=true,
    bibencoding=utf8,
    sortlocale=de_DE,
    style=numeric-comp,
    sorting=none
]{biblatex}
\addbibresource{bibliographie.bib}

\sisetup{locale = DE} 

\usepackage[nonumberlist,toc]{glossaries}
\displaywidowpenalty=10000
\graphicspath{{images/}}

\lstloadlanguages{PHP,Python,Java,C,C++,bash}

\lstset{literate=
  {á}{{\'a}}1 {é}{{\'e}}1 {í}{{\'i}}1 {ó}{{\'o}}1 {ú}{{\'u}}1
  {Á}{{\'A}}1 {É}{{\'E}}1 {Í}{{\'I}}1 {Ó}{{\'O}}1 {Ú}{{\'U}}1
  {à}{{\`a}}1 {è}{{\`e}}1 {ì}{{\`i}}1 {ò}{{\`o}}1 {ù}{{\`u}}1
  {À}{{\`A}}1 {È}{{\'E}}1 {Ì}{{\`I}}1 {Ò}{{\`O}}1 {Ù}{{\`U}}1
  {ä}{{\"a}}1 {ë}{{\"e}}1 {ï}{{\"i}}1 {ö}{{\"o}}1 {ü}{{\"u}}1
  {Ä}{{\"A}}1 {Ë}{{\"E}}1 {Ï}{{\"I}}1 {Ö}{{\"O}}1 {Ü}{{\"U}}1
  {â}{{\^a}}1 {ê}{{\^e}}1 {î}{{\^i}}1 {ô}{{\^o}}1 {û}{{\^u}}1
  {Â}{{\^A}}1 {Ê}{{\^E}}1 {Î}{{\^I}}1 {Ô}{{\^O}}1 {Û}{{\^U}}1
  {ã}{{\~a}}1 {ẽ}{{\~e}}1 {ĩ}{{\~i}}1 {õ}{{\~o}}1 {ũ}{{\~u}}1
  {Ã}{{\~A}}1 {Ẽ}{{\~E}}1 {Ĩ}{{\~I}}1 {Õ}{{\~O}}1 {Ũ}{{\~U}}1
  {œ}{{\oe}}1 {Œ}{{\OE}}1 {æ}{{\ae}}1 {Æ}{{\AE}}1 {ß}{{\ss}}1
  {ű}{{\H{u}}}1 {Ű}{{\H{U}}}1 {ő}{{\H{o}}}1 {Ő}{{\H{O}}}1
  {ç}{{\c c}}1 {Ç}{{\c C}}1 {ø}{{\o}}1 {å}{{\r a}}1 {Å}{{\r A}}1
  {€}{{\euro}}1 {£}{{\pounds}}1 {«}{{\guillemotleft}}1
  {»}{{\guillemotright}}1 {ñ}{{\~n}}1 {Ñ}{{\~N}}1 {¿}{{?`}}1 {¡}{{!`}}1 
}

\lstset{%
    language=Java,
    numbers=left,
    stepnumber=1,
    numbersep=5pt,
    numberstyle=\tiny,
    breaklines=true,
    breakautoindent=true,
    postbreak=\space,
    tabsize=4,
    basicstyle=\ttfamily\scriptsize,
    showspaces=false,
    showstringspaces=false,
    extendedchars=true,
    captionpos=b,
    backgroundcolor=\color{ListingBackground},
    xleftmargin=0pt,
    xrightmargin=0pt,
    frame=single,
    frameround=ffff,
    rulecolor=\color{darkgray},
    fillcolor=\color{ListingBackground},
    keywordstyle=\color[rgb]{0.133,0.133,0.6},
    commentstyle=\color[rgb]{0.133,0.545,0.133},
    stringstyle=\color[rgb]{0.627,0.126,0.941}
}
\renewcommand\lstlistingname{Quellcode}
\renewcommand\lstlistlistingname{Quellcodeverzeichnis}
\def\lstlistingautorefname{Quellcode}

\setlength{\tabcolsep}{10pt}
\renewcommand{\arraystretch}{1.1}

\numberwithin{equation}{section}

\usetikzlibrary{arrows}

\newcommand{\direktesZitat}[3]{
    \vspace*{0.8em}
    \begin{quotation}
        \textit{\enquote{#1}} \\
                      \\
        \textbf{#2} \\
        \scriptsize{#3}
    \end{quotation}
    \vspace*{0.55em}
}

\ifdefined\defEntwurf
\usepackage{background}
\usepackage{datetime2}
\SetBgContents{\texttt{Entwurf, Stand vom \DTMtoday{} \DTMcurrenttime}}
\SetBgScale{1.1}
\SetBgOpacity{1}
\SetBgColor{red}
\SetBgPosition{current page.south west}
\SetBgAnchor{right}
\SetBgHshift{0.7cm}
\SetBgVshift{-0.7cm}
\SetBgAngle{90}
\fi


\makeglossaries
%!TEX root = ../dokumentation.tex

\newglossaryentry{spectogram}{name={Spektrogramm},plural={Spektrogramme},description={Erklärung des Begriffes...}}
\newglossaryentry{dipswitch}{name={DIP-Schalter},description={Erklärung des Begriffes...}}

% Aufruf mit \gls{...}

% Es werden nur die Glossareinträge angezeigt, welche tatsächlich verwendet werden.
% Werden keine Glossareinträge verwendet, wird auch kein Glossar erzeugt!

\begin{document}

    % Deckblatt
    \begin{spacing}{1}
        %!TEX root = ../dokumentation.tex

\begin{titlepage}
    \begin{minipage}{6in}
        \centering
        \raisebox{-0.5\height}{\includegraphics[width=4.5cm]{images/dhbw.png}}
        \hspace*{4cm}
        \raisebox{-0.5\height}{\includegraphics[width=4.5cm]{images/firmenlogo.png}}
    \end{minipage}
    \enlargethispage{20mm}
    \begin{center}
        \vspace*{12mm}	{\LARGE\textbf \defTitel }\\
        \vspace*{12mm}	{\large\textbf \defArbeit}\\
        \vspace*{12mm}	im Studiengang \defStudiengang\\
    \vspace*{3mm}		an der \defHochschule\\
        \vspace*{12mm}  von\\
        \vspace*{3mm}   {\large\textbf{\defAutor}}\\
        \vspace*{12mm}  \defAbgabedatum\\
    \end{center}
    \vfill
    \begin{spacing}{1.2}
    \begin{tabbing}
        mmmmmmmmmmmmmmmmmmmmmmmmmm    \= \kill
        \textbf{Bearbeitungszeitraum}                      \> \defBearbeitungszeitraum\\
        \textbf{Matrikelnummer, Kurs}                      \> \defMatrikelnummer, \defKurs\\
        \textbf{Ausbildungsfirma}                          \> \defAusbildungsfirma\\
                                                           \> \defAusbildungsfirmaOrt\\
        \textbf{Betreuer der Ausbildungsfirma}             \> \defBetreuer\\
        \ifdefined\defGutachter\textbf{Gutachter der DHBW} \> \defGutachter\\\fi
    \end{tabbing}
    \end{spacing}
    \vspace*{15mm}
\end{titlepage}

    \end{spacing}
    \newpage

    \pagenumbering{Roman}

    % Sperrvermerk
    %!TEX root = ../dokumentation.tex

\thispagestyle{empty}

\section*{Sperrvermerk}
\vspace*{2em}

% Wortlaut des Sperrvermerks aus Anlage 1.1.13 Satz 3 der Prüfungsordnung der DHBW Mosbach:
% https://www.mosbach.dhbw.de/fileadmin/user_upload/dhbw/ressorts/pruefungsamt/34_2021_Bekanntmachung_StuPrO_Technik_inkl._Fuenfte_AEnderungssatzung.pdf

Der Inhalt dieser Arbeit darf weder als Ganzes noch in Auszügen Personen außerhalb des Prüfungsprozesses und des 
Evaluationsverfahrens zugänglich gemacht werden, sofern keine anderslautende Genehmigung vom Dualen Partner vorliegt.

\vspace{3em}

{\color{red}Ort und Datum der Unterschrift/Stempel}
\vspace{4em}

\rule{7cm}{0.4pt}\\
Name der Firma

    \newpage

    % Erklärung
    %!TEX root = ../dokumentation.tex

\thispagestyle{empty}

\section*{Erklärung}
\vspace*{2em}

Ich versichere hiermit, blablabla

\vspace{3em}

{\color{red}Ort und Datum der Unterschrift}
\vspace{4em}

\begin{tabular}{p{0.5\textwidth}}
    \rule{7cm}{0.4pt}\\
    Name des Studenten
\end{tabular}



    \newpage

    % Abstract
    \pagestyle{empty}

\section*{Abstract}

% TODO Englischsprachige Zusammenfassung

\newpage
\section*{Zusammenfassung}

% TODO Deutschsprachige Zusammenfassung

\newpage
\section*{Vorwort}

% Vorwort

    \newpage

    \pagestyle{plain} % nur Seitenzahlen im Fuß
    \RedeclareSectionCommand[beforeskip=20pt]{chapter} % stellt Abstand vor Kapitelüberschriften ein

    % Inhaltsverzeichnis
    \begin{spacing}{1.1}
        \begingroup
            % auskommentieren für Seitenzahlen unter Inhaltsverzeichnis
            \pagestyle{empty}
            \setcounter{tocdepth}{2}
            \tableofcontents
            \clearpage
        \endgroup
    \end{spacing}
    \newpage

    % Abkürzungsverzeichnis
    \addchap{Abkürzungsverzeichnis}

% WICHIG: Die längste Abkürzung (hier 'DHBW') muss als Parameter in Zeile 7 angegeben werden. Ansonsten
% kann es sein, dass die Darstellung des Abkürzungsverzeichnisses nicht richtig funktioniert.

\begin{acronym}[DHBW]
    \setlength{\itemsep}{0pt}
    \setlength{\parsep}{0pt}

    \acro{DHBW}{Duale Hochschule Baden-Württemberg}
    \acro{IT}{Informationstechnik}
    \acro{NAS}{Network Attached Storage}
    \acro{www}{World Wide Web}
\end{acronym}

% \ac{Abk.}   --> fügt die Abkürzung ein, beim ersten Aufruf wird zusätzlich automatisch die ausgeschriebene Version davor eingefügt
% \acs{Abk.}  --> fügt die Abkürzung ein
% \acf{Abk.}  --> fügt die Abkürzung UND die Erklärung ein
% \acl{Abk.}  --> fügt nur die Erklärung ein
% \acp{Abk.}  --> gibt Plural aus (angefügtes 's'); das zusätzliche 'p' funktioniert auch bei obigen Befehlen

    % Abbildungsverzeichnis
    \listoffigures

    %Tabellenverzeichnis
    \listoftables

    % Quellcodeverzeichnis
    \lstlistoflistings

    \newpage
    \pagenumbering{arabic}
    \pagestyle{headings} % Kolumnentitel im Kopf, Seitenzahlen im Fuß

    % Inhalt
    \foreach \i in {01,02,03,04,05,06,07,08,09,...,99} {%
        \edef\FileName{content/\i kapitel}%
            \IfFileExists{\FileName}{\input{\FileName}}{}
    }

    % Literaturverzeichnis
    \ifdefined\defAlleQuellenZeigen\nocite{*}\fi
    \printbibliography

    % Glossar
    \printglossary[style=altlist,title=Glossar]
    
    % sonstiger Anhang
    \addchap{Anhang}

\section*{A Name des ersten Anhanges}

\section*{B Name des zweiten Anhanges}
\end{document}
